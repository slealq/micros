\documentclass[12pt,letterpaper]{report} % Tipo de documento
%%%%%%%%%%%
% PREÁMBULO
%%%%%%%%%%%

% Paquetes de uso general
%%%%%%%%%%%%%%%%%%%%%%%%%%%%%%

% Para escribir tildes y eñes
\usepackage[utf8]{inputenc}                   		

% Para títulos de figuras y otros en español
\usepackage[spanish,es-noquoting]{babel} 
	% Cambiar nombre a tablas                  		
	\addto\captionsspanish{\renewcommand{\tablename}{Tabla}}
    % Cambiar nombre a lista de tablas
	\addto\captionsspanish{\renewcommand{\listtablename}{Índice de tablas}}
    % Cambiar nombre a capítulos
	%\addto\captionsspanish{\renewcommand{\chaptername}{}}	
    % Modificado por sleal 
    % Quitar el número 1, 2, 3 etc y dejar solo el nombre:
\usepackage{titlesec}
\titleformat
{\chapter} % command
[hang] % shape
{\bfseries\Huge} % format
{\thechapter} % label
{1ex} % sep
{
} % before-code
[
] % after-code

% Tamaño del área de escritura de la página
\usepackage{geometry}                         
\usepackage{float}	\geometry{left=18mm,right=18mm,top=21mm,bottom=21mm} 	

% Paquetes para matemáticas
%%%%%%%%%%%%%%%%%%%%%%%%%%%%%%

% Paquetes ams, desarrollados por la American Mathematical Society y mejoran la escritura de fórmulas y símbolos matemáticos.
\usepackage{amsmath}      
\usepackage{amsfonts}     	
\usepackage{amssymb}

% Paquetes para gráficas
%%%%%%%%%%%%%%%%%%%%%%%%%%%%%%


% Para insertar gráficas
\usepackage{graphicx}    

% Para colocar varias subfiguras
\usepackage[lofdepth,lotdepth]{subfig}	

% Lenguaje para producir gráficos vectoriales
\usepackage{tikz}

\usepackage{listings} 	% Utilizado para manejar código 

% Para crear circuitos eléctricos
\usepackage[american]{circuitikz}

% Paquetes de estilo y formato
%%%%%%%%%%%%%%%%%%%%%%%%%%%%%%

\definecolor{codegreen}{rgb}{0,0.6,0} % Las siguientes cuatro líneas son utilizadas para el formato del código.
\definecolor{codegray}{rgb}{0.5,0.5,0.5}
\definecolor{codepurple}{rgb}{0.58,0,0.82}
\definecolor{backcolour}{rgb}{0.95,0.95,0.92}

\lstdefinestyle{mystyle}{
    backgroundcolor=\color{backcolour},   
    commentstyle=\color{codegreen},
    keywordstyle=\color{magenta},
    numberstyle=\tiny\color{codegray},
    stringstyle=\color{codepurple},
    basicstyle=\footnotesize,
    breakatwhitespace=false,         
    breaklines=true,                 
    captionpos=b,                    
    keepspaces=true,                 
    numbers=left,                    
    numbersep=5pt,                  
    showspaces=false,                
    showstringspaces=false,
    showtabs=false,                  
    tabsize=2
}

\lstset{style=mystyle,
    inputencoding = utf8,  % Input encoding
    extendedchars = true,  % Extended ASCII
    texcl         = true,  % Activate LaTeX commands in comments
    %mathescape    = true   % Mathematical expressions between $
    captionpos    = b,     % Caption position
    literate      =        % Support additional characters
      {á}{{\'a}}1  {é}{{\'e}}1  {í}{{\'i}}1 {ó}{{\'o}}1  {ú}{{\'u}}1
      {Á}{{\'A}}1  {É}{{\'E}}1  {Í}{{\'I}}1 {Ó}{{\'O}}1  {Ú}{{\'U}}1
      {à}{{\`a}}1  {è}{{\`e}}1  {ì}{{\`i}}1 {ò}{{\`o}}1  {ù}{{\`u}}1
      {À}{{\`A}}1  {È}{{\'E}}1  {Ì}{{\`I}}1 {Ò}{{\`O}}1  {Ù}{{\`U}}1
      {ä}{{\"a}}1  {ë}{{\"e}}1  {ï}{{\"i}}1 {ö}{{\"o}}1  {ü}{{\"u}}1
      {Ä}{{\"A}}1  {Ë}{{\"E}}1  {Ï}{{\"I}}1 {Ö}{{\"O}}1  {Ü}{{\"U}}1
      {â}{{\^a}}1  {ê}{{\^e}}1  {î}{{\^i}}1 {ô}{{\^o}}1  {û}{{\^u}}1
      {Â}{{\^A}}1  {Ê}{{\^E}}1  {Î}{{\^I}}1 {Ô}{{\^O}}1  {Û}{{\^U}}1
      {œ}{{\oe}}1  {Œ}{{\OE}}1  {æ}{{\ae}}1 {Æ}{{\AE}}1  {ß}{{\ss}}1
      {ç}{{\c c}}1 {Ç}{{\c C}}1 {ø}{{\o}}1  {å}{{\r a}}1 {Å}{{\r A}}1
      {ñ}{{\~n}}1  {Ñ}{{\~N}}1  {¿}{{?`}}1  {¡}{{!`}}1
      % ¿ and ¡ are not correctly displayed if inconsolata font is used
      % together with the lstlisting environment. Consider typing code in
      % external files and using \lstinputlisting to display them instead.      
  }  


% Para cambiar la tipografía (si se desea)
% Open Sans
\usepackage[default,osfigures,scale=0.95]{opensans} %% 
\usepackage[T1]{fontenc}
% Gillius
%\usepackage[T1]{fontenc}
%\usepackage[default]{gillius}
% Epigrafica
%\usepackage{epigrafica}
%\usepackage[LGR,OT1]{fontenc}
% Alegreya
%\usepackage[T1]{fontenc}
%\usepackage[sfdefault]{AlegreyaSans} %% Option 'black' gives heavier bold face
%% The 'sfdefault' option to make the base font sans serif
%\renewcommand*\oldstylenums[1]{{\AlegreyaSansOsF #1}}

% Para la presentación correcta de unidades
\usepackage{siunitx}	  	

% Para hipervínculos y marcadores
\usepackage[colorlinks=true,urlcolor=blue,linkcolor=black,citecolor=green]{hyperref}     

% Para ubicar las tablas y figuras justo después del texto
\usepackage{float}	

% Para hacer tablas más estilizadas
\usepackage{booktabs}		

% Estilo de la bibliografía
\bibliographystyle{plain} 

% Estilo de la página
\pagestyle{plain} 

% Estilo de numeración
\pagenumbering{arabic}
\usepackage{lastpage}

% Para manejar los encabezados y pies de página
\usepackage{fancyhdr}
	% Contenido de los encabezados y pies de página
	\pagestyle{fancy}				
	\lhead{IE-0623 Microprocesadores}
	\chead{}
	\rhead{Proyecto Final}
	\lfoot{Escuela de Ingeniería Eléctrica}
	\cfoot{\thepage\ de \pageref{LastPage}}
	\rfoot{Universidad de Costa Rica}

\usepackage[utf8]{inputenc}
\begin{document}
\begin{titlepage}
\begin{center}

\textsc{\Large UNIVERSIDAD DE COSTA RICA}\\[2em]
\textsc{\Large Escuela de Ingeniería Eléctrica}\\[2em]
\textsc{\Large IE-0623 Microprocesadores}\\[5em]
%Figura


\vspace{4em}

\textsc{\huge \textbf{Proyecto Final}}\\[2em]
\textsc{\huge \textbf{RADAR 623}}\\[8em]


\textsc{Realizado por}\\[1em]


\textsc{ \large Stuart Leal Quesada B53777}\\[1em]


\end{center}

\vspace*{\fill}
\textsc{San José, Costa Rica \hspace*{\fill} 2019}


\end{titlepage}
\section*{Resumen}
Aquí va el resumen del trabajo escrito.
%\end{abstract}
\tableofcontents
\listoffigures
\listoftables
% Modificado por sleal 10/1/2017
%\chapter{Nomenclatura.}

% ============================================================================================================ ============================================================================================================
% ============================================================================================================ ============================================================================================================
% ============================================================================================================ ============================================================================================================
% ===================================================================================================== INTRODUCCIÓN ======================================================================================================
% ============================================================================================================ ============================================================================================================
% ============================================================================================================ ============================================================================================================

\chapter{Resumen}

La introducción va aquí.
\newpage



% ============================================================================================================ ============================================================================================================
% ============================================================================================================ ============================================================================================================
% ============================================================================================================ ============================================================================================================
% ===================================================================================================== DESARROLLO ========================================================================================================
% ============================================================================================================ ============================================================================================================
% ============================================================================================================ ============================================================================================================
\chapter{Desarrollo}

\section{Explicación general}


%%%%%%%%%%%%%%%%%%%%%%%%%%%%%%%%%%%%%%%%%%%%%%%%%% ATD_ISR %%%%%%%%%%%%%%%%%%%%%%%%%%%%%%%%%%%%%%%%%%%%%%%%%%%%%%%%%%%%%%%%%%%%%%%%%%%%%%%%%%%%%%%%%%%%%%%%
%%%%%%%%%%%%%%%%%%%%%%%%%%%%%%%%%%%
%%%%%%%%%%%%%%%%%%%%%%%%%%%%%%%%%%%%%%%%%%%%%%%%%%%%%%%%%%%%%%%%%%%%%%%%%%%%%%%%%%%%%%%%%%%%%%%%%%%%%%%%%%%%%%%%%%%%%%%%%%%%%%%%%%%%%%%%%%%%%%%%%%%%%%%%%%%
\newcommand{\subname}{ATD\_ISR}

\section{Subrutina \subname}

\subsection{Cálculos para interrupción \subname}
En lo que respecta a esta interrupción, debemos realizar los cálculos para encontrar el valor de Presescalador que vamos a utilizar. La fórmula que relaciona la frecuencia con el valor del preescalador es la siguiente:

\begin{equation}
    frs = \frac{BUS\_CLK}{2\cdot(PRS + 1)}
\end{equation}

Para este caso, queremos la frecuencia más baja posible para este periférico, por tanto sería utilizar el valor más alto de PRS posible. Lo cuál corresponde a:

\begin{equation}
    PRS = 31 = \$1F
\end{equation}

\subsection{Configuración de \subname}
Para el primer registro de control $ATD0CTL2$, queremos habilitar el módulo de conversiones con el bit $ADPU=1$, queremos habilitar la opción de $AFFC$, para que se borra la bandera de interrupción cuando se leen los registros de datos y también, habilitar las interrupciones con el bit $ASCIE=1$. Esto significa que:

\begin{align*}
    ATD0CTL2 = \$C2
\end{align*} 

Para el siguiente registro de control $ATDOCTL3$, queremos 6 conversiones, y además $FIFO=0$. Entonces tenemos que:

\begin{align*}
    ATD0CTL3 = \$30
\end{align*} 

En $ATD0CTL4$, vamos a querer configurar 4 periodos de reloj para el muestreo, y además $SRE8=1$ para tener conversiones a 8 bits. Además, vamos a querer el valor del preescalador en 31. Esto significa que:

\begin{align*}
    ATD0CTL4 = \$BF
\end{align*} 

Finalmente, tenemos que activar la justificación a la derecha, y las conversiones hacerlas sin signo ($DJM=1$ y $DSGN=0$). Además queremos configurar $SCAN=MULT=0$ para que se muestree sólo la entrada definida, 6 veces y se guarden los valores de $ADR0$ hasta $ADR5$. Y finalmente, queremos seleccionar la entrada 7 con los bits $CC$, $CB$ y $CA$. Entonces:

\begin{align*}
    ATD0CTL5 = \$87
\end{align*} 

\subsection{Vector de interrupción \subname}

Finalmente, el vector de interrupción para ATD se encuentra en la dirección $\$3E52$.


\subsection{Diagrama y explicación de la subrutina \subname}

%%%%%%%%%%%%%%%%%%%%%%%%%%%%%%%%%%%%%%%%%%%%%%%%%% TCNT_ISR %%%%%%%%%%%%%%%%%%%%%%%%%%%%%%%%%%%%%%%%%%%%%%%%%%%%%%%%%%%%%%%%%%%%%%%%%%%%%%%%%%%%%%%%%%%%%%%
%%%%%%%%%%%%%%%%%%%%%%%%%%%%%%%%%%%
%%%%%%%%%%%%%%%%%%%%%%%%%%%%%%%%%%%%%%%%%%%%%%%%%%%%%%%%%%%%%%%%%%%%%%%%%%%%%%%%%%%%%%%%%%%%%%%%%%%%%%%%%%%%%%%%%%%%%%%%%%%%%%%%%%%%%%%%%%%%%%%%%%%%%%%%%%%
\renewcommand{\subname}{TCNT\_ISR}


\section{Subrutina \subname}
\subsection{Cálculos para interrupción \subname}
\subsection{Configuración de \subname}
\subsection{Vector de interrupción \subname}
\subsection{Diagrama y explicación de la subrutina \subname}

%%%%%%%%%%%%%%%%%%%%%%%%%%%%%%%%%%%%%%%%%%%%%%%%%% CALCULAR %%%%%%%%%%%%%%%%%%%%%%%%%%%%%%%%%%%%%%%%%%%%%%%%%%%%%%%%%%%%%%%%%%%%%%%%%%%%%%%%%%%%%%%%%%%%%%%
%%%%%%%%%%%%%%%%%%%%%%%%%%%%%%%%%%%
%%%%%%%%%%%%%%%%%%%%%%%%%%%%%%%%%%%%%%%%%%%%%%%%%%%%%%%%%%%%%%%%%%%%%%%%%%%%%%%%%%%%%%%%%%%%%%%%%%%%%%%%%%%%%%%%%%%%%%%%%%%%%%%%%%%%%%%%%%%%%%%%%%%%%%%%%%%

\section{Subrutina CALCULAR}


\subsection{Diagrama y explicación de la subrutina TCNT\_ISR}

%%%%%%%%%%%%%%%%%%%%%%%%%%%%%%%%%%%%%%%%%%%%%%%%%% RTI_ISR %%%%%%%%%%%%%%%%%%%%%%%%%%%%%%%%%%%%%%%%%%%%%%%%%%%%%%%%%%%%%%%%%%%%%%%%%%%%%%%%%%%%%%%%%%%%%%%%
%%%%%%%%%%%%%%%%%%%%%%%%%%%%%%%%%%%
%%%%%%%%%%%%%%%%%%%%%%%%%%%%%%%%%%%%%%%%%%%%%%%%%%%%%%%%%%%%%%%%%%%%%%%%%%%%%%%%%%%%%%%%%%%%%%%%%%%%%%%%%%%%%%%%%%%%%%%%%%%%%%%%%%%%%%%%%%%%%%%%%%%%%%%%%%%

\section{Subrutina RTI\_ISR}

%%%%%%%%%%%%%%%%%%%%%%%%%%%%%%%%%%%%%%%%%%%%%%%%%% OC4_ISR %%%%%%%%%%%%%%%%%%%%%%%%%%%%%%%%%%%%%%%%%%%%%%%%%%%%%%%%%%%%%%%%%%%%%%%%%%%%%%%%%%%%%%%%%%%%%%%%
%%%%%%%%%%%%%%%%%%%%%%%%%%%%%%%%%%%
%%%%%%%%%%%%%%%%%%%%%%%%%%%%%%%%%%%%%%%%%%%%%%%%%%%%%%%%%%%%%%%%%%%%%%%%%%%%%%%%%%%%%%%%%%%%%%%%%%%%%%%%%%%%%%%%%%%%%%%%%%%%%%%%%%%%%%%%%%%%%%%%%%%%%%%%%%%

\section{Subrutina OC4\_ISR}


\subsection{Cálculos para interrupción RTI\_ISR}

Para esta subrutina, queremos que se ejecute cada 1ms. Entonces, tenemos la siguiente fórmula:

\begin{equation}
    T_{RTI} = \frac{(N+1)\cdot2^{M+9}}{OSC\_CLK}
\end{equation}

Haciendo un poco de retrospección, $OSC_CLK$ tiene un valor de $8MHz$, por lo que en el numerador necesitamos un número muy cercano a este valor. 

Recordando que $2^{13} = 8192 \approx 8\cdot10^3$, entonces tendríamos que:

\begin{align*}
    T_{RTI} = \frac{(0+1)\cdot2^{4+9}}{8\cdot10^6} = 1.024ms
\end{align*}

Entonces, con $N=1$ y $M=4$ logramos nuestro objetivo.

\subsection{Configuración de RTI\_ISR}

La configuración para este periferico es bastante sencilla en realidad. Básicamente, lo primero es configurar el tiempo de $T_{RTI}$ con los valores calculados anteriormente. Esto sería:

\begin{align*}
    RTICTL = \$40
\end{align*}

Lo segundo, habilitar el puerto de RTI. Esto último se hace con la siguiente configuración:

\begin{align*}
    CRGINT = \$80
\end{align*}

\subsection{Vector de interrupción RTI\_ISR}

El vector de interrupción se encuentra en la dirección $\$3E70$ para el Debug12.

\subsection{Cálculos para interrupción OC4\_ISR}

Tenemos que usar un Preescalador de 8 para la interrpución por overflow de $TCNT\_ISR$. Entonces, haciendo los cálculos para la cantidad de TICKS que tenemos que contar, para que $OC4\_ISR$ se ejecute con una frecuencia de $50kHz$ serían los siguiente: 

\begin{equation}
    TCS = \frac{20\mu s \cdot 24 MHz}{PRS}
\end{equation}

Esto significa entonces que, usando $PRS = 8$, encontramos que:

\begin{align*}
    TCS = 60
\end{align*}

Cada interrupción entonces, debemos cargar $TC4$ con $TCNT + 60$.

\textbf{NOTA:} Para realizar cuentas de $1ms$ por ejemplo, debemos contar hasta 50.

Cuentas de $100ms$ (Para llamar a $CONV\_BIN\_BCD$ y $BCD\_7SEG$) se realizan contando hasta 5000.

Cuentas de $200ms$ (Para llamar a $PATRON\_LEDS$ y $ATD0\_CTL5$) se realizan contando hasta 10000.


\subsection{Configuración de OC4\_ISR}

Para el primer registro de configuración $TSCR1$, queremos habilitar el módulo de Timers (bit $TEN$) y además, queremos habilitar la bandera de $TFFCA$. 

Esto quiere decir que, la bandera de $C4F$ se va a borrar cuando se escriba un dato en $TC4$, y además, la bandera de $TOF$ se va a borrar cuando se lea el registro de $TCNT$.

Entonces, para este primer registro, tenemos que:

\begin{align*}
    TSCR1 = \$90
\end{align*}

Luego tenemos el segundo registro de control, $TSCR2$, en donda vamos a configurar el preescalador con el valor de 8. Esto significa guardar un 3 en los bits de PRS. Por lo tanto, tenemos que:

\begin{align*}
    TSCR2 = \$03
\end{align*}

\textbf{NOTA}: Para habilitar o deshabilitar las interrupciones por rebase, la configuración se hace por este mismo registro, en el bit 7 del registro. Sin embargo no están habilitadas por defecto. Sólo se habilitan cuando se está en el modo medición.

Siguiendo con la configuración, para Output Compare, tenemos que habilitar la opción de output compare para el canal 4, en el registro de $TIOS$. Entonces, tenemos que:

\begin{align*}
    TIOS = \$10
\end{align*}

Y además, tenemos que habilitar la interrupción para cuando la bandera de $C4F$ se levanta. Esto se hace en el registro de configuración llamado $TIE$, en donde tenemos entonces que:

\begin{align*}
    TIE = \$10
\end{align*}

\subsection{Vector de interrupción OC4\_ISR}

El vector de interrupción para $OC4\_ISR$ se encuentra en la dirección $\$3E66$.

\subsection{Cálculos para TCNT\_ISR}
Para esta interrupción, lo que debemos de calcular es el tiempo que toma entre una interrupción y otra. Esto se puede calcular con la siguiente fórmula:

\begin{equation}
    T_{TOI} = \frac{2^{16} \cdot 8}{24MHz} = \frac{1024}{46875} seg
\end{equation}

\subsection{Configuración de TCNT\_ISR}

Para esta interrupción, la configuración del módulo de reloj ya se realizó previamente para $OC4\_ISR$. Para habilitar la interrupción en el modo medición, se hace escribiendo un 1 en el bit 7 del registro $TSCR2$. 

Para deshabilitar la interrupción, se escribe un 0 en el mismo bit 7 de ese registro.

\subsection{Vector de interrupción de TCNT\_ISR}

El vector de interrupción para esta subrutina se encuentra en la dirección $\$3E5E$.


\subsection{Cálculos para CALCULO - PH}

Para realizar el cálculo de la velocidad, usamos la siguiente fórmula:

\begin{equation}
    VELOC = \frac{40m}{n_{ticks}} \cdot \frac{tick}{seg} \cdot \frac{seg}{hora} \cdot \frac{km}{metro} 
\end{equation}

Desarrollando lo anterior, tenemos que:

\begin{align*}
    VELOC &= \frac{40}{n_{ticks}} \cdot \frac{46875}{1024} \cdot \frac{3600}{1} \cdot \frac{1}{1000} \\ \\
          &= \frac{25}{n_{ticks}} \cdot \frac{16875}{64} \\ \\ 
          &= \frac{421875}{n_{ticks} \cdot 64}  
\end{align*}
    
Entonces, la estrategia para realizar el cálculo de velocidad, será:

\begin{itemize}
    \item Realizar la multiplicación de $n_{ticks} \cdot 64$ y guardarlo en $X$.
    \item Cargar en $D$ $\#200$, luego en $Y$ $\#16785$. Multiplicación queda en $Y:D$.
    \item Dividir $Y:D$ entre $X$, y guardar el resultado en VELOC. 
\end{itemize}

\textbf{NOTA:} Hay que verificar que el resultado no sea más grande que $255$ (es posible si n es muy pequeño, como por ejemplo $n=1$ significa que $resultado = 6591$, lo cuál no es una velocidad con sentido. En caso de ser mayor a 255, guardar el máximo valor posible en VELOC (255).


\subsection{Configuración para PH}

Para hablitar todos los pines del puerto H como entradas, utilizamos el registro $DDRH$, escribiendo cero en todos los bits:

\begin{align*}
    DDRH = \$00
\end{align*}

Para habilitar las interrupciones para $PH3$ y $PH0$, utilizamos el registro $PIEH$:

\begin{align}
    PIEH = \$09
\end{align}

Para definir la activación con flanco decreciente, se pone en cero los bits 3 y 0 del registro $PPSH$:

\begin{align*}
    PPSH = \$F6
\end{align*}

Finalmente, para borrar todas las banderas de interrupción, en caso de que haya alguna en cola:

\begin{align*}
    PIFH = \$FF
\end{align*}

\subsection{Cálculos para PANT\_CTRL}
Lo primero, para lograr que $TICK\_DIS$ llegue a cero en dos segundos, entonces tenemos que cargar:

\begin{align*}
    TICK\_DIS = 2seg \cdot \frac{46875}{1024} = 91.55 \approx 92
\end{align*}

Ahora, en general, dado una velocidad $VELOC$, la cantidad de ticks que debemos cargar en  contar para que el carro haya avanzado una distancia en metros DISTANCIA, se calcula de la siguiente manera:

\begin{equation}
    TICKS = \frac{DISTANCIA}{VELOC} \cdot \frac{1km}{1000m} \cdot \frac{3600s}{1h} \cdot \frac{46875}{1024}
\end{equation}

Ahora, particularmente, para calcular $TICK\_EN$, la $DISTANCIA = 60m$. Entonces tendríamos lo siguiente:

\begin{align*}
    TICK\_EN &= \frac{100m \cdot 1km \cdot 3600 \cdot 46875}{VELOC \cdot 1000m \cdot 1h \cdot 1024} \\ \\
             &= \frac{16479.4921875}{VELOC} \\ \\ 
             &\approx \frac{16480}{VELOC}
\end{align*}

Por otro lado, para calcular $TICK\_DIS$, simplemente es el doble de $TICK\_EN$. Es decir:

\begin{align*}
    TICK\_DIS &= \frac{32958.984375}{VELOC} \\ \\
              &\approx \frac{32959}{VELOC}
\end{align*}


% ============================================================================================================ ============================================================================================================
% ============================================================================================================ ============================================================================================================
% ============================================================================================================ ============================================================================================================
% ===================================================================================================== CONCLUSIONES Y RECOMENDACIONES ====================================================================================
% ============================================================================================================ ============================================================================================================
% ============================================================================================================ ============================================================================================================
\chapter{Conclusiones y recomendaciones}
\section{Conclusiones}
\section{Recomendaciones}

%Agregar la bibliografía al indice

\renewcommand{\refname}{Bibliografía}
\addcontentsline{toc}{chapter}{Bibliografía}
\begin{thebibliography}{}

\bibitem{Art 1} Osorio M. (2011). \emph{Los robots basados en una arquitectura deliberativa y la toma de decisiones}. México: Revista saberes compartidos.

\bibitem{Art 2} Urdiales C. \& Bandera A. \& Sandoval S. (2014). \emph{Historia y tendencias actuales de la robótica}. España: Editorial Universidad de Málaga.

\bibitem{Art 3} Batle, J.A \& Barjau A. (2008). \emph{Holonomy in mobile robots}. España: Universidad de Catalunya, Barcelona. 

\bibitem{Art 4} Oliveira, H. P., Sousa, A. J., Moreira, A. P., \& Costa, P. J. (2009). Modelado de robots omnidireccionales de 3 y 4 ruedas. Contemporary Robotics: Challenges and Solutions.

\bibitem{Art 5} Carton Geek. (2016). Robot Omnidireccional. [online] Cartongeek.blogspot.com. Available at: http://cartongeek.blogspot.com/2016/02/robot-omnidireccional.html [Accessed 20 May 2017].

\bibitem{Art 6} Muñoz V. \& Gil-Gómez G. \& García A. \emph{Modelado cinemático y dinámico de un robot móvil omni-direccional}. Málaga: Universidad de Málaga.

\bibitem{Art 7} Barrero L. \& Villegas A. \& Gómez D.
\emph{Robot transportador y distribuidor de objetos según su peso}. Redes ingeniería. Volumen 5.


\bibitem{Art 8} Martínez S. \& Sisto R. \emph{Proyecto de Grado: Control y Comportamiento de Robots Omnidireccionales}. Instituto de Computación. Facultad de Ingeniería - Universidad de la República
Montevideo. Available at: https://www.fing.edu.uy/inco/grupos/mina/pGrado/easyrobots/doc/SOA.pdf[Accessed 05 June 2017]
\bibitem{Art 9} García D.(2012) 
\emph{Modelado y Simulacion de un Robot Autónomo Omnidireccional}. Universidad Auntónoma de Querétaro. 
\bibitem{Art 10} Barrero L. \& Villegas A. \& Gómez D. (2014)
\emph{Robot Transportador Omnidireccional }. AMDM 2014.

\bibitem{Art 11} Ramos E. \& Morales R. \& Silva R. (2010) \emph{Modelado, simulación y construcción de un robot móvil de ruedas tipo diferencial}. México: CIDETEC.

\bibitem{Art 14} Rojas R. (2005) \emph{Omnidirectional Control}. Alemania: Freie Universitat Berlin.

\bibitem{Art 12} Suárez A. \& Sánchez A. \emph{Plataforma Móvil omnidireccional de cuatro llantas suecas (Mecanum) en configuración AB}.  México: Universdad Autónoma de México.

\bibitem{Art 13} V. F. Muñoz Martínez, \emph{Modelado cinemático y dinámico de un robot móvil omnidireccional}. Instituto Andaluz de Automática Avanzada y Robótica.

\bibitem{Art 16} Autonomy Lab of Simon Fraser University, \emph{Moving or Sensing, Time and Energy}. Vancouver, Canada.

\bibitem{Art 15}  Cornwell J. \emph{Mecánica Vectorial para Ingenieros DINÁMICA} Novena Edición.

\bibitem{Art 17} Williams, R., Carter, B. , Gallina,  P., \& Rosati, G. (2002). \emph{Dynamic Model With Slip for Wheeled Omnidirectional Robots}. IEEE TRANSACTIONS ON ROBOTICS AND AUTOMATION, VOL. 18, NO. 3, JUNE 2002. 


\end{thebibliography}
% Agregar la línea para los apéndices
\addcontentsline{toc}{chapter}{Apéndices}



\end{document}
